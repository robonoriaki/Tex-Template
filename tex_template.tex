%	UTF8
\documentclass [titlepage, a4j, 11pt] {jsarticle}

\listfiles
\usepackage [dvipdfmx]{graphicx}
\usepackage{slashbox}
\usepackage {bm}
\usepackage {amssymb}
\usepackage {amsmath}
%\usepackage {showkeys}
\usepackage {comment}
\usepackage {here}	%Hを使うためのパッケージ
%\usepackage{ascmac} %itembox, screenで角が丸くなる
\usepackage {fancyhdr}
\pagestyle{fancy}

\lhead{}%left header
%\lhead{\rightmark}
\chead{}%center header
\rhead{}%right header
%\rhead{\leftmark}
\def\headrulewidth{0pt}%header line 

 \lfoot{}%left footer
\cfoot{\thepage}%center footer
\rfoot{}%right fooer

\def\footrulewidth{0pt}%footer line 


%\graphicspath{{../fluid_analysis_pictures/}}%グラフのパス

%%%%%%%%%%%%%%%%%%%%%%%%%%%%%%%%%%%%%%%%%%%%%%%%%%%%%%%%%%%%%%%%%%%%%
%% 高さの設定
\setlength {\textheight} {\paperheight}   % ひとまず紙面を本文領域に
\setlength {\topmargin} {-5.4truemm}      % 上の余白を20mm(=1inch-5.4mm)に
\addtolength {\topmargin} {-\headheight}  % 
\addtolength {\topmargin} {-\headsep}     % ヘッダの分だけ本文領域を移動させる
\addtolength {\textheight} {-40truemm}    % 下の余白も20mmに
%% 幅の設定
\setlength {\textwidth} {\paperwidth}     % ひとまず紙面を本文領域に
\setlength {\oddsidemargin} {-5.4truemm}  % 左の余白を20mm(=1inch-5.4mm)に
\setlength {\evensidemargin} {-5.4truemm} % 
\addtolength {\textwidth} {-40truemm}     % 右の余白も20mmに

%図,表の表示名
\renewcommand {\figurename} {Fig.}
\renewcommand {\tablename} {Table}

%図と表を横に並べるためのもの
\makeatletter
\newcommand{\figcaption}[1]{\def\@captype{figure}\caption{#1}}
\newcommand{\tblcaption}[1]{\def\@captype{table}\caption{#1}}
\makeatother
%この時,表のcaptionは\tblcaptionとする

%図,表,式などの間隔
\setlength {\abovecaptionskip} {1mm}	%図・表とキャプションの間隔の変更
\setlength {\belowcaptionskip} {1mm}
\setlength {\abovedisplayskip} {3pt} % 式の上部のマージン
\setlength {\belowdisplayskip} {3pt} % 式の下部のマージン

%図番号を(subsection).(図番号)に変更
\makeatletter
\renewcommand{\thefigure}{\thesection.\arabic{figure}}
\@addtoreset{figure}{section}

%表番号を(subsection).(表番号)に変更
\renewcommand{\thetable}{\thesection.\arabic{table}}
\@addtoreset{table}{section}

%式番号を(subsection).(式番号)に変更
\renewcommand{\theequation}{\thesection.\arabic{equation}}
\@addtoreset{equation}{section}
\makeatother

%注釈
%\renewcommand\thefootnote{*\arabic{footnote}}

%目次の表示レベル設定
\setcounter {tocdepth} {3}

\usepackage{listings}%プログラムのソースコードを挿入するためのもの

\lstset{
  language={C},
  basicstyle={\small},%
  identifierstyle={\small},%
  %commentstyle={\small\itshape},%
  commentstyle={\small},%
  keywordstyle={\small\bfseries},%
  ndkeywordstyle={\small},%
  stringstyle={\small\ttfamily},
  frame={tb},
  breaklines=true,
  columns=[l]{fullflexible},%
  numbers=left,%
  xrightmargin=0zw,%
  xleftmargin=3zw,%
  numberstyle={\scriptsize},%
  stepnumber=1,
  numbersep=1zw,%
  lineskip=-0.5ex%
}

%%%%%%%%%%%%%%%%%%%%%%%%%%%%%%%%%%%%%%%%%%%%%%%%%%%%%%%%%%%%%%%%%%%%%%%
\begin {document}

\begin {titlepage}
	\begin {center}
		\vspace {20mm}
		{\Large Class name} \\
		\vspace {5mm}
		%{\Large 卒業論文} \\
		\vspace {60mm}
		{\huge Title} 
		
%{}
%{ aaaaaaaaaaaaaaaaaaaaaaaaaaaaaaaaaaaa}
%{}\\
		\vspace {100mm}
		{\Large Date} \\
		\vspace {15mm}
	
	{\Large University} \\
	{\Large Faculty} \\
	{\Large Department} \\
	%\vspace {5mm}
	{\Large Student number} \\
	{\Large Name} \\
	\end {center}
\end {titlepage}

%アブストラクト(英語)
%\renewcommand {\abstractname} {\LARGE Abstract}
%\include {include/abstract_en}
%アブストラクト(日本語)
%\renewcommand {\abstractname} {\LARGE 要旨}
%\include {include/abstract_jp}

%sectionをchapterみたいに変更
%\makeatletter
%\renewcommand{\presectionname}{第} %sectionの数字前に第を挿入
%\renewcommand{\postsectionname}{章} %sectionの数字後に章を挿入
%\renewcommand{\section}{%
%    \if@slide\clearpage\fi
%    \@startsection{section}{1}{\z@}%
%    {\Cvs \@plus.5\Cdp \@minus.2\Cdp}% 前アキ
%    {.5\Cvs \@plus.3\Cdp}% 後アキ
%    {\normalfont\LARGE\headfont\raggedright}}


%目次
%\pagenumbering {Roman}		%ページ番号をローマ数字に
%\tableofcontents

%\newpage

% 図目次の表示
%\listoffigures
% 表目次の表示
%\listoftables

%%%%%%%%%%%%%%%%%%%%%%%%%%%%%%%%%%%%%%%%%%%%%%%%%%%%%%%%%%%%%%%%%%%%%%%%%%%%%%%%%%%%%%%%%%%%%%%%%%%%%%%%%%%%%%%%%%
%\clearpage
\setcounter{page}{1}
\pagenumbering{arabic}	%ページ番号をアラビア数字に
\section{Inrtoduction}
\begin{eqnarray}
y_1&=&f_1(x_1,x_2,\cdots,x_6) \nonumber \\
y_2&=&f_2(x_1,x_2,\cdots,x_6)\nonumber \\
\vdots \\
y_6&=&f_6(x_1,x_2,\cdots,x_6)\nonumber 
\end{eqnarray}

\begin{equation}
Y=F(X)
\end{equation}

\begin{equation}
X=
	\begin{bmatrix}
	x_1 &x_2 &x_3 &x_4 &x_5 &x_6
	\end{bmatrix}
^T
\end{equation}

\begin{equation}
Y=
	\begin{bmatrix}
	y_1 &y_2 &y_3 &y_4 &y_5 &y_6
	\end{bmatrix}
^T
\end{equation}

\begin{eqnarray}
\delta y_1&=&\cfrac{\partial f_1}{\partial x_1}\delta x_1+\cfrac{\partial f_1}{\partial x_2}\delta x_2+\cdots+
				\cfrac{\partial f_1}{\partial x_6}\delta x_6 \nonumber \\
\delta y_2&=&\cfrac{\partial f_2}{\partial x_1}\delta x_1+\cfrac{\partial f_2}{\partial x_2}\delta x_2+\cdots+
				\cfrac{\partial f_2}{\partial x_6}\delta x_6 \nonumber \\
\vdots \\
\delta y_6&=&\cfrac{\partial f_m}{\partial x_1}\delta x_1+\cfrac{\partial f_m}{\partial x_2}\delta x_2+\cdots+
				\cfrac{\partial f_6}{\partial x_6}\delta x_6 \nonumber
\end{eqnarray}

\newpage
\section{Experiment}
\ref{fig:mani1}
%%図を入れる
\begin{figure}[H]
\begin{minipage}{0.5 \hsize}
	\begin{center}
		\includegraphics[keepaspectratio,width=80mm]{manipulator-2.png} %picture name and size
		\caption{マニピュレータが伸びきったときの特異点}
		\label{fig:mani1}
	\end{center}
\end{minipage}
\begin{minipage}{0.5 \hsize}
	\begin{center}
		\includegraphics[keepaspectratio,width=80mm]{manipulator2-2.png} %picture name and size
		\caption{マニピュレータが自身の方に折りたたまれたときの特異点}
		\label{fig:mani2}
	\end{center}
\end{minipage}
\end{figure}

\begin{figure}[H]
	\begin{center}
		\includegraphics[keepaspectratio,width=60mm]{wrist_singularity.png} %picture name and size
		\caption{手首特異点}
		\label{fig:sin3}
	\end{center}
\end{figure}

\end {document}